\documentclass{article}
\usepackage[utf8]{inputenc}
\usepackage[T1]{fontenc}
\usepackage{graphicx}
\usepackage{a4wide}
\usepackage[T1]{fontenc}
\usepackage{natbib}	
\usepackage{subcaption}
\usepackage{float}
\usepackage[english, frenchb]{babel}

\usepackage{flafter}
\usepackage{threeparttable}
\makeatletter
\newcommand\primitiveinput[1]
{\@@input #1 }
\makeatother
\usepackage{booktabs}
\usepackage{tikz}
\usepackage{enumitem}

\title{Modélisation des trajectoires indiciaires: \\ Méthodologie et résultats}

\author{Mahdi Ben Jelloul, Lise Degalle, Simon Rabaté}

\begin{document}


\maketitle


Dans le cadre du développement du modèle de microsimulation des retraites du régime des fonctionnaires territoriaux et hospitaliers, réalisé au sein de la Caisse des dépôts, l'Institut des politiques publiques contribue à la réalisation du module \og rémunération \fg{}. Cette note a pour but de décrire la méthodologie employée, et de présenter les résultats obtenus. 

%A la suite d'une introduction présentant de manière synthétique les résultats obtenus, nous présentons de manière détaillée la constitution de la base utilisée. Nous présentons ensuite les résultats empiriques sur l'effet des contraintes institutionnelles des grilles sur les trajectoires indiciaires. Enfin, nous présentons les résultats des simulations de trajectoires d'indice à partir de la modélisation adoptée. 

\tableofcontents


\clearpage

\section{Résumé de l'étude}


\section{Les données}

L'approche adoptée pour la modélisation des trajectoires est la suivante: nous simulons les changements de grade et l'évolution au sein du grade, en fonction de variables telles que la durée passée dans le grade ou l'échelon courant. Nous justifions empiriquement cette approche dans la partie suivante. Dans cette partie nous présentons l'échantillon sur lequel ont été réalisées les estimations et simulations. 

Comme mentionné en introduction, les travaux ont été réalisés sur une sous-population spécifique, les fonctionnaires territoriaux du corps des adjoints techniques. Ce sélection de l'échantillon s'accompagne de restrictions supplémentaires qui sont 


\subsection{Sélection de l'échantillon}


\subsection{Étape de complétion}


\subsection{Description de l'échantillon}


\section{Impact des grilles sur les changements de grade}

\subsection{Corps}

\subsection{Impact des conditions}

\subsection{Analyse économétrique}



\section{Simulation des trajectoires indiciaires}

\subsection{Méthodologie}

\subsection{Résultats}



\appendix

\section{Complétion des durées initiales}


\section{Simulation de l'évolution indiciaire}

\section{Liste des scripts transmis}

\paragraph{extract_data.py}



%%%%%%%%%%%%%%%%%%%%%%%%%%%%%%%%%%%%% Conclusion %%%%%%%%%%%%%%%%%%%%%%%%%%%%%%%%%%%%%%%%%%%%%%%%%%%%%%%%%%%%%%%%%%%%%%%
%\section{Conclusion}



\end{document}