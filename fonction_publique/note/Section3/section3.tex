\ifx\isEmbedded\undefined


\documentclass[11pt,a4paper]{article}
\usepackage[utf8]{inputenc}		% LaTeX, comprend les accents !
\usepackage[T1]{fontenc}
\usepackage{natbib}	
%\usepackage[square,sort&compress,sectionbib]{natbib}		% Doit être chargé avant babel      
\usepackage[frenchb,english]{babel}
\usepackage{lmodern}
\usepackage{amsmath,amssymb, amsthm}
\usepackage{a4wide}
\usepackage[capposition=top]{floatrow}
\usepackage{verbatim}
\usepackage{float}
\usepackage{placeins}
\usepackage{flafter}
\usepackage{longtable}
\usepackage{pdflscape}
\usepackage{rotating}
\usepackage{hhline}
\usepackage{multirow}
\usepackage{booktabs}
\usepackage[pdftex,pdfborder={0 0 0},colorlinks=true,linkcolor=blue,urlcolor=blue,citecolor=blue,bookmarksopen=true]{hyperref}
\usepackage{eurosym}
\usepackage{breakcites}
\usepackage[autostyle]{csquotes}
%\usepackage{datetime}
\usepackage{natbib}
\usepackage{setspace}
\usepackage{lscape}
\usepackage[usenames]{color}
\usepackage{indentfirst}

\usepackage{forest}
\usepackage{url}
\usepackage{enumitem}
\usepackage{multirow}
\usepackage{subcaption}
\usepackage[justification=centering]{caption}
\bibliographystyle{agsm}

\usepackage{array}

\begin{document}

\else \fi
%%%%%%%%%%%%%%%%%%%%%%%%%%%%%%%%%%%%%%%%%%%%%%%%%%%%%%%%%%%%%%%%%%%%%%%%%%%%%%%%%%%%%%%%%%%%%%%%%%%%%%%%%%%%%%



\section{Les modélisations économétriques envisagées}

Comme précisé précédemment, trois processus distincts doivent être modélisés: 
\begin{enumerate}[leftmargin=1cm ,parsep=0cm,itemsep=0cm,topsep=0cm]
\item La vitesse de franchissement des échelons au sein d'un grade
\item Le passage au grade supérieur dans le corps (\textit{a priori}, quand l'individu arrive en fin de grille)
\item Les mouvements plus importants: changement de corps, de catégorie hiérarchique, de FP (\textit{a priori}, pouvant intervenir n'important quand)
\end{enumerate}

Dans cette section nous mentionnons les pistes envisagées pour la modélisation de ces différents processus. Il s'agit de pistes de réflexions engagées avant une étude approfondie des données, et ne sont donc pas définitives. 


\subsection{Modéliser la vitesse dans le grade avec des effets fixes?}

Par vitesse dans le grade nous entendons la durée passée dans les échelon successifs d'un grade donné. Nous n'avons pas d'idée \textit{a priori} sur l'ampleur du phénomène. La dimension législative est sans doute importante: la progression dans l'échelon est parfois contrainte par une durée minimale et une durée maximale, voire une durée fixe pour certains grades. 

Un première question sera donc de savoir si cette dimension doit être modélisée (dans quelle mesure observe-t-on de la dispersion interindividuelle sur ce point?), et comment peut-on prendre en compte les législations spécifiques à chaque grade. 

La modélisation envisagée est la suivante: chaque individu possède une vitesse relative propre (un effet fixe), qui détermine la durée passée dans l'échelon relativement aux autre individus du grade. L'approche par effets fixes se justifient car, \textit{a priori}, la modélisation de la vitesse dans l'échelon est nécessaire uniquement si ce sont les mêmes individus qui franchissent plus rapidement les échelons au cours de leur carrière. L'idée est donc d'identifier des individus plus ou moins \og rapides \fg{} dans leur progression. Ces effets fixes se rapprochent des effets fixes dans les équations de salaires usuellement utilisées dans les modèles de microsimulation.  
A chaque date $t$, l'individu franchit ou non l'échelon en fonction de (i) sa durée passée dans l'échelon (ii) la durée \og normale \fg{} dans l'échelon et (iii) sa vitesse relative de franchissement des échelons. 

Certains tests rapides pourront être mis en \oe uvre pour attester de la pertinence d'une telle approche. Une fois reconstituée une trajectoire professionnelle de long terme, nous pourrons vérifier que les individus qui franchissent relativement plus rapidement un échelon donné, le font également sur l'ensemble de la grille considérée d'une part, et sur les autres grilles sur lesquelles on l'observe d'autre part. L'idée étant de vérifier que l'on peut modéliser la vitesse relative par un terme fixe au cours de la carrière. 


\subsection{Modéliser le \og choix \fg{} en fin de grille par un modèle de choix discret?}

Quand l'individu arrive en fin de grille (par hypothèse, ce qui peut se discuter\footnote{Étant donné que l'on observe des chevauchement dans les indices bruts des grilles d'un même corps, nous pouvons envisager que la décision du changement de grille peut intervenir avant l'arrivée au dernier échelon. Le choix du moment où l'on modélise la décision du passage au grade supérieur est conditionné par l'analyse de la temporalité des changement de grade, devant être menée en amont. }), il peut faire face à différents \og choix \fg{}: (i) passer dans le (ou les) grades supérieurs dans leur corps, (ii) rester à l'échelon maximal du grade ou (iii) quitter le corps, voire la fonction publique. 

Cette problématique suggère une modélisation du phénomène comme \og choix discrets \fg\ : il s'agirait d'estimer la probabilité d'un ensemble de choix, non ordonnés.
C'est une modélisation répandue notamment en économie du travail, lorsqu'on cherche à estimer la probabilité d'être en emploi à temps plein, en emploi à temps partiel,
en recherche d'emploi, actif inoccupé, \textit{etc.} L'estimation se fait par logit multinomial.



\paragraph{L'hypothèse d'indépendance des hypothèses non pertinentes}\footnote{\textit{IIA: Independance from Irrelevant alternatives}} Une limite importante de ce type de modélisation est que, le logit multinomial étant une généralisation du modèle binaire, on suppose que le choix entre deux alternatives est indépendant des autres alternatives. Plus précisément, la probabilité de choisir une alternative par rapport à une deuxième n'est pas conditionné par le contenu d'une troisième alternative. 

L'exemple classique est le choix d'un mode de transport, entre le vélo, le bus et la voiture. La probabilité de choisir la voiture plutôt que le vélo n'est pas indépendante de la qualité du service de bus: une amélioration de la qualité conduit à augmenter la probabilité de choisir le vélo par rapport à la probabilité de choisir la voiture, car la première n'est \textit{a priori} pas affecté par la qualité du bus mais la deuxième l'est. Une solution usuelle est d'utiliser des logits emboités, regroupant les choix potentiellement corrélés. On décomposera le choix vélo, voiture et bus en un choix entre vélo et véhicule motorisé, choix lui même décomposé entre un choix entre voiture et bus. 

%Cette hypothèse ne semble pas problématique dans le cas de l'exemple donné, mais si on souhaite aussi modéliser le congé maladie, la mise en disponibilité ou le détachement, l'hypothèse est moins crédible.

Comment cela se transpose-t-il au cas étudié? Supposons qu'il y a 4 alternatives possibles pour un individu se trouvant au dernier échelon du grade 1 d'une fonction publique donnée: (i) aller dans le grade 2, (ii) aller dans le grade 3, (iii) rester dans le grade 1, et (iv) quitter la fonction publique (ou changer de fonction publique). \textit{A priori}, l'hypothèse IIA n'est pas respectée pour les alternatives (i) et (iii). Si le grade 3 devient relativement moins attractif, on peut supposer que la probabilité de (iv) n'est pas affectée alors que la probabilité de (i) pourrait l'être car le contenu du grade 1 est sans doute plus proche du contenu du grade 2. Une première étape évidente serait donc d'utiliser un logit emboîté rassemblant les différents  grades potentiels au sein de la FP dans lequel l'individu se trouve. Nous nous ramenons donc à trois issues potentielles (i) être promu au sein de la FP considérée, (ii) ne pas être promus et rester (iii) ne pas être promu et quitter le grade, le (i) pouvant être redécomposé en différents grades possibles. 
L'hypothèse IIA est-elle vérifiée dans ce cas ? Dans le choix entre (i) et (ii), une modification des perspectives extérieures à la FP considérée (par exemple un concours de changement de FP) devrait plus affecter les individus qui seraient restés dans le grade que les individus qui sont promus quoiqu'il arrive. Il semble donc également pertinent de regrouper les alternatives (ii) et (iii) dans le cadre d'un logit emboité. 

Nous sommes donc ramenés à un double logit emboité. 



\tikzset{
Above/.style={
  midway,
  above,
  font=\scriptsize,
  text width=1.5cm,
  align=center,
  },
Below/.style={
  midway,
  below,
  font=\scriptsize,
  text width=1.5cm,
  align=center
  }
}

\begin{center}
\begin{forest} 
for tree={
  grow=east,
  draw=cyan,
  circle,
  line width=0.2pt,
  parent anchor=east,
  child anchor=west,
  edge={draw=cyan},
  edge label={\Huge\color{black}},
  edge path={
    \noexpand\path[\forestoption{edge}]
      (!u.parent anchor) -- ([xshift=-2cm].child anchor) --    
      (.child anchor)\forestoption{edge label};
  },
  l sep=2cm,
} 
[,rectangle, s sep=35pt,
  [,edge label={node[Below]{Pas de promotion}}
    [,edge label={node[Below]{Quitter la FP considérée}}
    ]
    [,edge label={node[Above]{Rester dans grade 1}}
    ]
  ]
  [,edge label={node[Above]{Promotion dans la FP considérée}}
    [,edge label={node[Below]{Grade 3}}
    ]
    [,edge label={node[Above]{Grade 2}}
    ]
  ]
]
\end{forest}

\end{center}

\vspace{0.5cm}

Une autre possibilité serait de modéliser les sorties du corps en fin de grille de manière séparée, avec les changements de grade en cours de grille, évoqués ci-dessous. La modélisation représentée ci-dessus revient à faire l'hypothèse implicite que les changement de corps en fin de grille sont davantage liés à un manque de perspective dans le corps (individus bloqués en fin de grille), alors que les changement de corps en cours de grille sont plus \og positifs \fg{}. Mais il s'agit à ce stade de pure spéculation. Nous pourrons tester dans une certaine mesure cette hypothèse quand nous pourrons comparer les types de changement de grades en fin de grille et sur l'ensemble de la grille. 


\subsection{Modéliser les changement de grade en cours de grille: modèle de durée, choix discret, ou tirage aléatoire?}

A tout moment, les fonctionnaires peuvent changer de statut (de grade, de catégorie, de corps), par exemple en passant un concours. Nous considérons à part ce type de mouvement, en les considérant comme une déviation par rapport au parcours défini par l'évolution \og naturelle \fg{}  au sein des grades et des corps. 

Ce type de processus est potentiellement complexe à modéliser, car les destinations possible quand un individu quitte son grade sont nombreuses: changement de grade au sein d'un corps, changement de catégorie, changement de fonction publique, sortie de la fonction publique. C'est d'ailleurs sur ce type de processus que les interactions avec les modules affiliations et carrières sont les plus difficiles à gérer. 

La complexité de la modélisation doit être comparée à l'importance du phénomène: si le processus représente une part faible des évolutions observées, une modélisation relativement sommaire est envisageable. Ainsi par exemple à chaque pas temporelle une certaines proportion d'individus peut être tirés puis envoyé vers un grade différent (avec pour seule contrainte un indice brut supérieur). Cette modélisation peut-être affiné dans le cadre d'un modèle binaire (logit ou probit) faisant dépendre la probabilité de changer de grade de certaines caractéristiques observables. 

Si ce processus est plus important qu'envisagé \textit{a priori}, une modélisation plus fine devra être adoptée. Deux grands type d'approche sont considérés à ce stade. Premièrement, des modèles à choix discret peuvent être testés A chaque date, un fonctionnaire fait face à un ensemble de choix: rester sur son \og tapis roulant \fg{}, passer un concours, partir dans le privé, etc. Un tel modèle peut en théorie être estimé sur les données passées, éventuellement après regroupement des alternatives proches. Nous envisageons également un modèle de durée, avec différents états concurrents, modélisant la survie dans le grade. A chaque date, l'individu a une probabilité de quitter le grade dans lequel il est, selon certaines caractéristiques observables (niveau dans le grade), et éventuellement inobservables en prenant en compte les durées passées observée dans les grades ou échelons précédents (\textit{multispell duration models}). 


\bigskip

L'analyse de la temporalité des changements de grade est donc un préalable indispensable à la poursuite de la réflexion sur la modélisation de la rémunération. Les modélisations proposées reposent sur des représentations \textit{a priori} des trajectoires indiciaires qui doivent être confrontées aux données. En particulier, notre analyse est structurée autour de la distinction entre les changements de grade \og naturels \fg{} (en fin de grille, au sein du corps, vers les grades supérieurs normaux) et plus imprévisibles (à tout moment et a priori plus importants: changement de corps, de catégorie, de fonction publique).  

Les questions à traiter en priorité sont les suivantes: 
\begin{itemize}
    \item Comment les changements de grades sont-ils repartis dans la grille? Sont-ils concentrés sur les derniers échelons ?
    \item Les changements de fin d'échelon sont-ils bien plus prévisibles que les autres changements observés ?     
\end{itemize}

Des réponses à ces questions dépendent la pertinence des choix de modélisations envisagés.

%%%%%%%%%%%%%%%%%%%%%%%%%%%%%%%%%%%%%%%%%%%%%%%%%%%%%%%%%%%%%%%%%%%%%%%%%%%%

\ifx\isEmbedded\undefined
\newpage
\bibliographystyle{../../Divers/myagsm} 
\bibliography{../../Divers/biblio_these}
\end{document}
\else \fi

