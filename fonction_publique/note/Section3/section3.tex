\ifx\isEmbedded\undefined


\documentclass[11pt,a4paper]{article}
\usepackage[utf8]{inputenc}		% LaTeX, comprend les accents !
\usepackage[T1]{fontenc}
\usepackage{natbib}	
%\usepackage[square,sort&compress,sectionbib]{natbib}		% Doit être chargé avant babel      
\usepackage[frenchb,english]{babel}
\usepackage{lmodern}
\usepackage{amsmath,amssymb, amsthm}
\usepackage{a4wide}
\usepackage[capposition=top]{floatrow}
\usepackage{verbatim}
\usepackage{float}
\usepackage{placeins}
\usepackage{flafter}
\usepackage{longtable}
\usepackage{pdflscape}
\usepackage{rotating}
\usepackage{hhline}
\usepackage{multirow}
\usepackage{booktabs}
\usepackage[pdftex,pdfborder={0 0 0},colorlinks=true,linkcolor=blue,urlcolor=blue,citecolor=blue,bookmarksopen=true]{hyperref}
\usepackage{eurosym}
\usepackage{breakcites}
\usepackage[autostyle]{csquotes}
%\usepackage{datetime}
\usepackage{natbib}
\usepackage{setspace}
\usepackage{lscape}
\usepackage[usenames]{color}
\usepackage{indentfirst}

\usepackage{url}
\usepackage{enumitem}
\usepackage{multirow}
\usepackage{subcaption}
\usepackage[justification=centering]{caption}
\bibliographystyle{agsm}

\usepackage{array}

\begin{document}

\else \fi
%%%%%%%%%%%%%%%%%%%%%%%%%%%%%%%%%%%%%%%%%%%%%%%%%%%%%%%%%%%%%%%%%%%%%%%%%%%%%%%%%%%%%%%%%%%%%%%%%%%%%%%%%%%%%%



\section{Modélisation économétrique}

\subsection{Les changements de grade: logit multinomial}

Nous avons vu %à vérifier :)
que le classement hiérarchique des agents de la fonction publique n'est pas exactement un \og tapis roulant \fg, et qu'entre autres, lorsqu'un agent arrive en fin de grade, 
le grade de \og destination \fg\ varie. Pour pouvoir projeter les carrières des agents, il faut donc estimer la probabilité de passer dans tel ou tel grade après avoir fait
une carrière dans tel autre grade.

Cette problématique évoque une modélisation du phénomène comme \og choix discrets \fg\ : il s'agirait d'estimer la probabilité d'un ensemble de choix, non ordonnés.
C'est une modélisation répandue notamment en économie du travail, lorsqu'on cherche à estimer la probabilité d'être en emploi à temps plein, en emploi à temps partiel,
en recherche d'emploi, actif inoccupé, \textit{etc.} L'estimation se fait par logit multinomial.

Une limite importante de ce type de modélisation est que, le logit multinomial étant une généralisation du modèle binaire, on suppose que le choix entre deux alternatives
(par exemple, passer dans le grade X ou Y après avoir été dans le grade V) est indépendant des autres alternatives (le passage au grade Z après avoir été dans le grade V).
Cette hypothèse ne semble pas problématique dans le cas de l'exemple donné, mais si on souhaite aussi modéliser le congé maladie, la mise en disponibilité ou le détachement,
l'hypothèse est moins crédible.% : ce qui rend la mise en disponibilité plus attractive rend probablement moins souhaitable de passer dans le grade X, Y ou Z.
% Non c'est pas ça. Expliquer pourquoi

-> Logits emboîtés

+ Mixed logit : pour intégrer des coefficients pour certains choix mais pas pour d'autres. (cf. cours de Luc)


\subsection{Le parcours dans le grade: un modèle dynamique ?}





Question: quel degré de finesse dans le choix des sous-catégories? 
Par FP, par corps? par catégorie hiérarchique? 



%%%%%%%%%%%%%%%%%%%%%%%%%%%%%%%%%%%%%%%%%%%%%%%%%%%%%%%%%%%%%%%%%%%%%%%%%%%%

\ifx\isEmbedded\undefined
\newpage
\bibliographystyle{../../Divers/myagsm} 
\bibliography{../../Divers/biblio_these}
\end{document}
\else \fi

