 
\documentclass[11pt,a4paper]{article}
\usepackage[utf8]{inputenc}		% LaTeX, comprend les accents !
\usepackage[T1]{fontenc}
\usepackage{natbib}	
%\usepackage[square,sort&compress,sectionbib]{natbib}		% Doit être chargé avant babel      
\usepackage[frenchb,english]{babel}
\usepackage{lmodern}
\usepackage{amsmath,amssymb, amsthm}
\usepackage{a4wide}
\usepackage[capposition=top]{floatrow}
\usepackage{verbatim}
\usepackage{float}
\usepackage{placeins}
\usepackage{flafter}
\usepackage{longtable}
\usepackage{import}
\usepackage{pdflscape}
\usepackage{rotating}
\usepackage{hhline}
\usepackage{multirow}
\usepackage{booktabs}
\usepackage[pdftex,pdfborder={0 0 0},colorlinks=true,linkcolor=blue,urlcolor=blue,citecolor=blue,bookmarksopen=true]{hyperref}
\usepackage{eurosym}
%\usepackage{breakcites}
\usepackage[autostyle]{csquotes}
%\usepackage{datetime}
\usepackage{natbib}
\usepackage{setspace}
\usepackage{lscape}
\usepackage[usenames]{color}
\usepackage{indentfirst}
\usepackage{url}
\usepackage{enumitem}
\usepackage{multirow}
\usepackage{subcaption}
\usepackage[justification=centering]{caption}
\bibliographystyle{agsm}

\usepackage{array}

\newcommand{\isEmbedded}{true}

\graphicspath{{../bordeaux/results/}}


\begin{document}

\selectlanguage{frenchb}
\title{Compléter les carrières en rétrospectif}


\author{Simon Rabat\'e}


\maketitle

% Introduction
La modélisation des trajectoires indiciaires des agents de la CNRACL nécessite de pouvoir
observer, sur une durée relativement importante, l’évolution des carrières à la CNRACL.
Concrètement, cela implique d’être en mesure de connaitre, à chaque date t, la position exacte
de l’individu dans les grilles indiciaires : le versant, le corps, le grade et in fine l’échelon dans
ce grade.
Le problème est que l’échelon n’est jamais directement observé, il faut donc l’imputer à
partir des informations disponibles dans les données. A partir du grade, de la date, et de
l’indice correspondant, il est normalement possible d’obtenir l’échelon correspondant. Outre
le fait que ceci doit encore être vérifié dans les données, la variable de grade n’est disponible
qu’à partir de l’année 2011. L’étude des trajectoires professionnelles nécessite donc une étape
de complétion des grades à partir des libellés saisis à la main, disponible dans la base carrière
à partir de 2000.

% Section I: principe général 
\section{Données disponibles, données manquantes}

La reconstitution des trajectoires dans les grilles repose sur deux sources de données :
\begin{itemize}[leftmargin=2cm ,parsep=0cm,itemsep=0cm,topsep=0cm] 
\item Les grilles de la fonction publiques dématérialisées. Pour chaque grille, nous disposons
des informations suivantes : la date d'effet, l'indice brut correspondant à chaque échelon, et les conditions de durée passée dans le grade (durée min, durée max, ou durée moyenne).
\item La base carrière, qui comporte des informations sur les trajectoires individuelles dans le régime. Les variables disponibles varient en fonction des années considérées
(cf. infra).

\end{itemize}

A partir de ces deux sources d'information, l'objectif est d'être en mesure de replacer
chaque point observé dans la base carrière dans les grilles correspondantes, pour recomposer
les trajectoire au sein des grilles (ou au sein du corps, du versant, ou de la FP en général).
Comme mentionné en introduction, l'échelon est en théorie défini de manière unique par
le grade et l'indice brut. En pratique, des discordances peuvent exister, en cas d'erreur de
saisie sur l'une des variables ou de l'absence de prise en compte de changement de grille par
les agents. Par ailleurs la définition du grade donné dans les grilles (grade neg) ne correspond
pas forcément à la variable de grade que l'on observe dans les données (cf. note de PJ du 24
juin 2016 et section 2). Une première étape de complétion consiste donc à faire correspondre
le grade observé dans la base carrière au grade neg permettant de retrouver l'échelon à partir
des grilles.
De manière plus problématique, les variables de grades (CIR ou NETNEH) ne sont renseignés
que pour les années 2011 à 2014. Obtenir des trajectoires indiciaires sur une plage
temporelle plus importante implique donc une étape de complétion supplémentaire, pour obtenir
le grade à partir des variables disponibles. En particulier, nous utilisons une variable
de libellés de grade saisi à la main (variable libemploi, disponible à partir de 2010). Un programme
de matching du libellé rempli à la main au code grade neg a été développé par l'IPP,
visant à attribuer à chaque épisode d'emploi dans le régime un grade (section \ref).


% Section II: data
\section{Du libellé grade NETNEH au grade NEG}



% Section III: Modélisation
\section{Du libellé au grade : une procédure de matching}



% Section IV: Simulation
\section{Du grade à l'échelon}

\subsection{Grade $\bigcap$ indice = échelon}


\section*{Liste de questions}

\begin{itemize}[leftmargin=1cm ,parsep=0cm,itemsep=0cm,topsep=0cm] 
\item Qui remplit les libellés saisis à la main dans la base carrière (variable libemploi) ? Cela
se fait-il au niveau de la fiche de paie ou du calcul des droits à la retraite ? Cela se fait-il
sur la base du code NEG ou du code NETNEH?
\item Dans quelle mesure est-ce un problème d'utiliser l'indice pour attribuer le grade à partir
du libellé puis d'utiliser l'indice pour déterminer l'échelon à partir du grade attribué ?
\item Pourquoi les variables issues des CIR sont-elles disponibles sur des années différentes
(code grade et libellé grade).
\item Faut-il conserver les individus/observation avec libelles ou code manquant
\end{itemize}



\end{document}


