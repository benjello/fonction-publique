\documentclass[xcolor=table,ignorenonframetext,12pt]{beamer}
%\documentclass[xcolor=table,handout,12pt]{beamer}
\usepackage[frenchb]{babel}
\usepackage[utf8]{inputenc}
\usepackage{amsmath,amssymb}
\usepackage{graphicx}
\usepackage{pgfarrows,pgfnodes}
\usepackage{url}
\usepackage{textcomp}
\usepackage[vcentermath]{youngtab}
\usepackage{epstopdf}
\usepackage{hhline}
\usepackage{xmpmulti}
\usepackage{arcs}                       % Pour réaliser le logo TAXIPP
%\usepackage{slashbox}                   % Pour réaliser le logo TAXIPP
\usepackage{MnSymbol}                   % Pour réaliser le logo TAXIPP
\PassOptionsToPackage{usenames,dvipsnames,svgnames}{xcolor} % Pour réaliser le logo TAXIPP
\usepackage{xcolor}                     % Pour réaliser le logo TAXIPP
\usepackage{multirow}
\usepackage{subfigure}
\usepackage{booktabs}
\usepackage{tikz}
%\usepackage{bbold}
\usepackage{changepage}
\mode<article> {
  \usepackage{fullpage}
  \usepackage{pgf}
  \usepackage{hyperref}
}
\usepackage{tikz}
\usetikzlibrary{shapes.geometric, arrows, automata,positioning}

\mode<presentation>
%\usetheme{Madrid}
%\usetheme[left]{Goettingen}
%\setbeamertemplate{background canvas}{\includegraphics[width=\paperwidth]{ippheader.pdf}}
\setbeamertemplate{background canvas}{\hspace{10.1cm}\includegraphics
    [width=0.2\paperwidth]{logo_ipp.pdf}}
\definecolor{ippdark}{RGB}{0,93,116}
\definecolor{ipplight}{RGB}{0,142,156}
\definecolor{ippxlight}{RGB}{0, 204, 204}
\useinnertheme[shadow=true]{rounded}
\setbeamertemplate{items}[circle]
\setbeamercolor{title}{bg=ipplight, fg=black}
\setbeamercolor{structure}{fg=ipplight}
%\setbeamertemplate{sidebar canvas left}{} % pour supprimer le fond de couleur de la barre latérale	
\setbeamertemplate{sidebar canvas left}[vertical shading][top=structure.fg!20,bottom=structure.fg!15]
\setbeamertemplate{caption}[numbered]
\setlength{\leftmargini}{12pt}
\setbeamerfont{framesubtitle}{size=\large}
\definecolor{ippdark}{RGB}{0,93,116}
\definecolor{ipplight}{RGB}{0,142,156}

%\usepackage[latin1]{inputenc}
%\usepackage[english]{babel}

% danger logo
\newcommand*{\TakeFourierOrnament}[1]{{%
\fontencoding{U}\fontfamily{futs}\selectfont\char#1}}
\newcommand*{\danger}{\TakeFourierOrnament{66}}

\newenvironment{checklist}[1]{\begin{list}{$\surd$}{}#1}{\end{list}}%


\newlength{\offsetpage}
\setlength{\offsetpage}{1.0cm}
\newenvironment{widepage}{\begin{adjustwidth}{-\offsetpage}{-\offsetpage}%
		\addtolength{\textwidth}{2\offsetpage}}%
	{\end{adjustwidth}}


% Macro d'inclusion de graphiques pdf %
\newcommand{\graphique}[2][1]{\begin{minipage}{\linewidth}\begin{center}\includegraphics[width=#1\linewidth,clip]{Graphiques/#2}\end{center}\end{minipage}}

\newcommand{\os}[2]{\onslide+<#1->{#2}}%

\newcommand{\m}[2]{\multicolumn{#1}{c}{#2}}%
\newcommand{\ml}[2]{\multicolumn{#1}{l}{#2}}%
\newcommand{\tth}{\textsuperscript{th}}%
\newcommand{\nde}{\textsuperscript{nde}}
\newcommand{\er}{\textsuperscript{er}}
\newcommand{\eme}{\textsuperscript{ème}~}
\newcommand{\hligne}{\begin{tikzpicture}[remember picture,overlay]\node[shift={(-10.5 cm,-1.2cm)}]at(current page.north east){\begin{tikzpicture}{\draw[line width=0.2mm,color=ipplight,overlay](0,0)--(8,0);}\end{tikzpicture}};\end{tikzpicture}\vspace{-0.8cm}}
\newcommand{\hlignee}{\begin{tikzpicture}[remember picture,overlay]\node[shift={(-10.5 cm,-1.2cm)}]at(current page.north east){\begin{tikzpicture}{\draw[line width=0.2mm,color=ipplight,overlay](0,0)--(8,0);}\end{tikzpicture}};\end{tikzpicture}}

% MTABLE: macro for tables %
\newenvironment{mfigure}[4][1]{\def\TMP{#3}\newdimen\TMPsize\settowidth{\TMPsize}{\TMP}
\begin{figure}\caption{#2}\begin{center}\begin{tiny}
\begin{minipage}{#1\textwidth}\resizebox{\textwidth}{!}{#3}\end{minipage}
\if!#4!\empty \else \\
\resizebox{#1\textwidth}{!}{\begin{minipage}{\TMPsize}\begin{tiny}\smallskip\par
#4 \end{tiny}\end{minipage}} \fi}
{\end{tiny}\end{center}\end{figure}}

% Macro pour créer un tableau (avec notes optionnelles) %
\newenvironment{tab}[4][1]
{\def\TMP{#3}\newdimen\TMPsize\settowidth{\TMPsize}{\TMP}
\begin{table}[!ht]
\begin{center}
\begin{minipage}{14cm}
\caption{#2}
\end{minipage}
\end{center}
\begin{center}
\begin{minipage}{#1}
\resizebox{\textwidth}{!}{#3}
\end{minipage}
\if!#4!\empty \else \\
\begin{scriptsize}
\begin{minipage}{#1}\vspace{0.2cm} \par #4
\end{minipage}
\end{scriptsize} \fi }
{\end{center}
\end{table}}

\newenvironment{choixmarges}[2]{\begin{list}{}{%
\setlength{\topsep}{0pt}%
\setlength{\leftmargin}{0pt}%
\setlength{\rightmargin}{0pt}%
\setlength{\listparindent}{\parindent}%
\setlength{\itemindent}{\parindent}%
\setlength{\parsep}{0pt plus 1pt}%
\addtolength{\leftmargin}{#1}%
\addtolength{\rightmargin}{#2}%
}\item }{\end{list}}


\beamertemplatenavigationsymbolsempty


\makeatletter
\def\PENSIPP{PENS\kern-.05em\lower-.19ex\hbox{${\color{BlueGreen} \scalebox{1.4}{\underarc[1]{\overarc[1]{\textcolor{black}{\scalebox{0.7}{ipp}}}}}}\hspace{0.5ex}$}\@}
\makeatother

% numérotation de slides
\setbeamertemplate{footline}[frame number]
%
\AtBeginSection[]
{
  \begin{frame}[noframenumbering]
    \frametitle{\large Outline}
    \small \tableofcontents[currentsection,hideothersubsections, subsectionstyle=hide]
  \end{frame}
}


\title{Modélisation des trajectoires dans le grade : premières simulations}
\author{Simon Rabaté*, Mahdi Ben Jelloul* \& Lisa Degalle*}
\institute{
  \inst{*} IPP
}
\subject{}

\setbeamercolor{alerted text}{fg=green}

\date{CoPil CNRACL\\
	Paris, 17 juillet 2017}

\begin{document}


\frame{\maketitle}


%\begin{frame}
%    \frametitle{Plan de la présentation}
%    \tableofcontents[hidesubsections]
%\end{frame}



\section{Introduction}


\begin{frame}
\frametitle{Introduction}
\framesubtitle{Rappel du point d'étape précédent}


\begin{choixmarges}{-0.5cm}{-0.5cm}



\begin{itemize}
\item \textbf{Objectif du module :} modéliser la trajectoire indiciaire des fonctionnaires
\vspace{0.1cm}


\item \textbf{Résultats:}  


\begin{itemize}

\item Imputation de la durée dans le grade: 

\begin{itemize}
\item Utilisation des trajectoires indiciaires avant 2011
\end{itemize}

\item Impact important des conditions institutionnelles
\begin{itemize}
\item Statistiques descriptives
\item Estimations
\end{itemize}

\item Peu d'effet en simulations
\begin{itemize}
\item Modèle avec des polynômes de durée aussi bon
\end{itemize}

\end{itemize}


\vspace{0.1cm}

\item\textbf{Prochaines étapes} prévues pour juillet:
\begin{itemize}
\item Choix multiples (logit multinomial)
\item Modélisation des trajectoires en échelon
\end{itemize}


\end{itemize}



\end{choixmarges}
\end{frame}



\begin{frame}
\frametitle{Ordre du jour 1 : }
\framesubtitle{Poursuite du travail d'estimation/simulation}

\begin{choixmarges}{-0.5cm}{-0.5cm}
\begin{itemize}
\item Objectif : $\mathbf{ib_t \rightarrow ib_{t+1}}$ ($\rightarrow ib_{t+2} \rightarrow ib_{t+3}$ ...)

\vspace{0.2cm}
\item Nouveaux résultats:
\begin{itemize}
\item Logit multinomial: modélisation de la destination 
\begin{enumerate}
\item Reste dans le grade
\item Grade suivant dans le corps
\item Grade hors du corps
\end{enumerate}
\item Attribuer un ib en fonction de la situation en $t+1$ 
\begin{itemize}
\item Reste dans le grade : changement échelon?
\item Grade suivant : quel échelon?
\item Grade hors du corps: quel grade? quel échelon?
\end{itemize}
\end{itemize}

\vspace{0.2cm}
\item Points à améliorer à court terme
\begin{itemize}
\item Méthodologie d'estimation: nested logit? afe?
\item Règles de détermination de $ib_{t+1}$
\item Tests d'adéquation
\end{itemize}

\end{itemize}

\end{choixmarges}
\end{frame}




\begin{frame}
\frametitle{Ordre du jour 2 : }
\framesubtitle{Points à discuter} 


\begin{choixmarges}{-0.5cm}{-0.5cm}


\begin{enumerate}

\item Stratégie empirique
	\begin{itemize}
	\item Logit vs. modèle de durée
	\item Choix des variables explicatives
	\item Tests d'adéquation
	\end{itemize}
	
\vspace{0.2cm}	
\item Mise en \oe uvre dans le modèle
	\begin{itemize}
	\item Généralisation de l'approche
	\item Langage(s) de programmation
	\item Articulation avec les autres modules
	\item Travail sur les données
		\begin{itemize}
		\item Imputation du grade en retrospectif
		\item Correction des trajectoires aberrantes 
		\end{itemize}
	\end{itemize}

\vspace{0.2cm}
\item Simuler des réformes de grilles
	\begin{itemize}
	\item Quel type de réforme? Quel effet attendu? 
	\end{itemize}



\end{enumerate}

\end{choixmarges}
\end{frame}



%%%%%%%%% Section 1 -  The completion of the career trajectories %%%%%%%%%%%
\section{Stratégie empirique}


\begin{frame}
\frametitle{Approche globale}



\tikzstyle{startstop} = [rectangle, rounded corners, minimum width=0.5cm, minimum height=0.5cm,text centered, draw=black, fill=ipplight]
\tikzstyle{io} = [trapezium, trapezium left angle=70, trapezium right angle=110, minimum width=1cm, minimum height=1cm, text centered, draw=black, fill=ippdark]
\tikzstyle{process} = [rectangle, minimum width=0.5cm, minimum height=0.5cm, text centered, draw=black, fill=ippxlight]
\tikzstyle{decision} = [diamond, minimum width=0.5cm, minimum height=0.5cm, text centered, draw=black, fill=ippdark]
\tikzstyle{arrow} = [thick,->,>=stealth]

\begin{widepage}
\begin{center}
%\resizebox{1\textwidth}{3.cm}{
\begin{tikzpicture}[node distance=5cm, sibling distance = 7cm]

\node (start) [startstop, xshift=-3cm, yshift=-0.4cm] {$ib_t$ = ($C_t, G_t, E_t$)};
\node (choice)[decision, below  of=start,  yshift=3cm] {\shortstack{Sortie\\grade?}};

\node (in1) [decision, left of=choice, xshift=1cm]  {\shortstack{Sortie\\corps?}};

\node (in2) [process, below left  of=in1, xshift=0cm, yshift=1.5cm] {\shortstack{Choix \\ du grade}};

\node (in3) [process,  below of=in1,  xshift=0.5cm, yshift=1.5cm] {\shortstack{Echelon du \\nouveau grade ?}};	
	
\node (in4) [process, below right  of=choice, xshift=-1cm, yshift=0cm] {\shortstack{Changement\\échelon?}}
	;

\node (stop)  [startstop, below  of=choice , yshift=-0.5cm] {$ib_{t+1}$ = ($C_{t+1}, G_{t+1}, E_{t+1}$)};

\draw[arrow] (start) -- (choice);
\draw[arrow] (choice)  -- node[anchor=south]{oui}(in1);
\draw[arrow] (in1)  -- node[anchor=east]{oui}(in2);
\draw[arrow] (in2) -- (in3);
\draw[arrow] (in1)  -- node[anchor=west]{non}(in3);
\draw[arrow] (choice) -| node[anchor=south]{non}(in4);
\draw[arrow] (in3) -- (stop);
\draw[arrow] (in4) -- (stop);



\end{tikzpicture}

%}

\end{center}

\end{widepage}

\end{frame}





\begin{frame}
\frametitle{Déterminer le grade en t+1}
\begin{itemize}
\item Estimation d'un logit multivarié avec trois options possible: 
	\begin{enumerate}
	\item Rester dans le grade
	\item Sortir du grade vers le grade suivant dans le corps
	\item Sortir du grade vers un grade dans un autre corps
	\end{enumerate}

\vspace{0.1cm}	
\item Population d'estimation:
 	\begin{itemize}
	\item Estimation sur 2011 seulement 
	\item Trajectoires propres (cf. CoPil précédent)
	\item Suppression des trajectoires avec sortie vers autre grade et interruption d'activité
 	\end{itemize}

\vspace{0.1cm}		
\item Variables explicatives:
     \begin{itemize}
     \item Fixes: génération, genre, grade
     \item Variables: durée dans le grade, seuils institutionnels
     \end{itemize}

\vspace{0.1cm}	
\item Utilisation des résultats:
     \begin{itemize}
     \item probabilité d'être dans un état donné à la date donnée selon les caractéristiques individuelles
     \end{itemize}


\end{itemize}


\end{frame}



\begin{frame}
\frametitle{Prédiction de l'échelon en t+1}
\begin{itemize}
\item Deux méthodes distinctes selon que l'agent \textbf{change de grade} ou qu'il \textbf{reste dans son grade}
\end{itemize}

	\textbf{1.} L'agent change de grade : \\
	\begin{enumerate}
	\item S'il change de corps, son $Grade_{t+1}$ prédit est celui vers lequel le plus grand nombre d'agents changeant de corps transitionnent sur la période 2011-2015 depuis $Grade_{t}$\\
	\begin{footnotesize}
		\begin{itemize}
		\item Ce grade est le grade TTM1 pour tous les grades des ATT 
		\item On pourrait plutôt tirer un $Grade_{t+1}$ dans la distribution empirique des grades d'accueil des agents changeant de corps depuis $Grade_{t}$
	\end{itemize}
	\end{footnotesize}
	\item On le place ensuite à l'échelon \textbf{correspondant à l'IB immédiatement supérieur ou égal} à son $IB_t$ sur la grille du $Grade_{t+1}$ 
	\end{enumerate}
	
	

\end{frame}

\begin{frame}
\frametitle{Prédiction de l'échelon en t+1}


	\textbf{2.}  L'agent ne change pas de grade : il avance d'échelon selon la \textbf{durée minimale} d'avancement donnée par les grilles\\
	\begin{itemize}
		\item Les réformes de ces durées minimales sont prises en compte
\begin{itemize}
					\begin{footnotesize}
			\item[] $\Rightarrow$ Les agents bénéficiant d'une réforme raccourcissant la durée légale qu'ils doivent passer dans l'échelon de telle sorte qu'ils ont déjà rempli cette nouvelle condition au moment de la réforme passent à leur échelon supérieur au moment de la réforme
					\end{footnotesize}
\end{itemize}
		\item Les agents parvenus au dernier échelon de leur grille y restent jusqu'à la fin de la période de simulation désirée
		\item La procédure est rapide et fonctionne quel que soit le grade
	\end{itemize}

	

\end{frame}

\begin{frame}
\frametitle{Prédiction de l'échelon : test}

\begin{figure}
	\vspace{-0.5cm}
	\includegraphics[scale=0.8]{Graphiques/hist_echelon.pdf}
\end{figure}


\end{frame}



\section{Résultats}


\begin{frame}
\frametitle{Résultats}


\textbf{Résultats présentés}
\begin{enumerate}
\item Estimation du logit multivarié
\item Tests d'adéquation des simulations
\begin{enumerate}
\item[Test 1:] Prédictions des départs entre 2011 et 2014
\item[Test 2:] Comparaison $ib^{sim}$ et $ib^{obs}$ en 2012 
\end{enumerate} 
\end{enumerate}


\vspace{0.2cm}
\textbf{Résumé des résultats}
\begin{itemize}
\item Confirmation des résultats précédents:
\begin{itemize}
\item Effet important des seuils pour le passage au grade suivant
\item Le modèle avec la durée uniquement fait presque aussi bien 
\end{itemize}

\item Résultat encore provisoires:
\begin{itemize}
\item Amélioration de l'évolution dans le grade (échelon)
\item Règle de l'attribution de l'échelon quand changement de grade à affiner
\end{itemize}

\end{itemize}



\end{frame}




\begin{frame}
\frametitle{Résultats des estimations}

\begin{table}[!ht]
\begin{center}
\scriptsize
\begin{tabular}{l c }
\toprule
 & Model 1 \\
\midrule
exit\_next:(intercept)                      & $-4.547^{***}$ \\
exit\_oth:(intercept)                       & $-3.599^{***}$ \\
\textbf{exit\_next:I\_unique\_threshold  }           & $\mathbf{1.813^{***}}$ \\
\textbf{exit\_oth:I\_unique\_threshold  }            & \textbf{$\mathbf{0.339^{***}}$}  \\
exit\_next:c\_cir\_2011TTH2                 & $2.015^{***}$  \\

exit\_oth:c\_cir\_2011TTH2                  & $1.485^{***}$  \\
 
exit\_next:c\_cir\_2011TTH3                 & $1.451^{***}$  \\
  
exit\_oth:c\_cir\_2011TTH3                  & $1.766^{***}$  \\
exit\_next:c\_cir\_2011TTH4                 & $-19.873$      \\
    
exit\_oth:c\_cir\_2011TTH4                  & $1.649^{***}$  \\
                                        
exit\_next:sexeM                            & $0.414^{***}$  \\
exit\_oth:sexeM                             & $-0.757^{***}$ \\
exit\_next:duration\_bef\_unique\_threshold & $0.072^{***}$  \\
exit\_oth:duration\_bef\_unique\_threshold  & $0.069^{***}$  \\
exit\_next:duration\_aft\_unique\_threshold & $-0.015^{*}$   \\
exit\_oth:duration\_aft\_unique\_threshold  & $-0.041^{***}$ \\
\midrule
Num. obs.                                   & 89302          \\
\bottomrule
\multicolumn{2}{l}{\scriptsize{$^{***}p<0.01$, $^{**}p<0.05$, $^*p<0.1$}}
\end{tabular}
\label{table:coefficients}
\end{center}
\end{table}


\end{frame}


\begin{frame}
\frametitle{Test d'adéquation 1}
\framesubtitle{Prédiction des départs entre 2011 et 2014}
\begin{figure}
	\vspace{-0.5cm}
	\includegraphics[scale=0.5]{Graphiques/surv1.pdf}
\end{figure}

\end{frame}


\begin{frame}
\frametitle{Test d'adéquation 1}
\framesubtitle{Prédiction des départs entre 2011 et 2014}
\begin{figure}
	\vspace{-0.5cm}
	\includegraphics[scale=0.5]{Graphiques/surv2.pdf}
\end{figure}

\end{frame}


\begin{frame}
\frametitle{Test d'adéquation 2}
\framesubtitle{Comparaison des ib prédits et observés}

\begin{itemize}

\item Quelles métriques utiliser? 
\item A ce stade: 
\begin{enumerate}
\item Agrégats: masse des ib en 2012
\item Comparaison des distribution en 2012
\end{enumerate}

\vspace{0.2cm}
\item Résultats:
\begin{enumerate}
\item Bonne prédiction pour les modèles m1 et m2 
\begin{itemize}
\item 1.5\% d'erreur sur la masse des ib totale
\item 7\% en cas de changement de grade
\end{itemize}
\item Sous estimation systématique des ib prédit 
\begin{enumerate}
\item En l'absence de changement de grade \\ (cf. problème d'initialisation).
\item Quand changement de grade: affiner la règle d'attribution du grade et de l'échelon.
\end{enumerate}

\end{enumerate}

\end{itemize}

\end{frame}


\begin{frame}
\frametitle{Test d'adéquation 2}
\framesubtitle{Comparaison des ib prédits et observés}

\begin{figure}
	\caption{Distributions des ib en cas de sortie prédite}
		\vspace{-0.3cm}
	\includegraphics[scale=0.5]{Graphiques/distr_ib.pdf}
\end{figure}

\end{frame}


\section{Questions à aborder}





\begin{frame}
\frametitle{Points à discuter}
\framesubtitle{Stratégie empirique}

\begin{choixmarges}{-0.5cm}{-0.5cm}


\begin{itemize}


\item Logit vs. modèle de durée
	\begin{itemize}
	\item Abandon modèle de durée pour l'instant
	\begin{itemize}
	\item Manque de profondeur temporelle
	\item Reprise avec le matching?
	\end{itemize}
	
	\item Logit vs. nested logit ?
	\begin{itemize}
	\item IIA respectée? 
	\item Regrouper les sorties de grade.
	\end{itemize}
	\end{itemize}

\vspace{0.2cm}
\item Choix des variables explicatives
	\begin{itemize}
	\item Ajouter des variables lié à l'historique de carrière? 
	\begin{itemize}
	\item Arrêts maladies? 
	\item Proxy de l'âge de fin d'étude (\textit{via} DAI)
	\end{itemize}
	\end{itemize}

\vspace{0.2cm}
\item Tests d'adéquation à définir ensemble

\end{itemize}


\end{choixmarges}
\end{frame}




\begin{frame}
\frametitle{Points à discuter}
\framesubtitle{Mise en \oe uvre dans le modèle}

\begin{choixmarges}{-0.5cm}{-0.5cm}


\begin{itemize}

\item Généralisation de l'approche
	\begin{itemize}
	\item Autres grades de la FPT
	\item FPH
	\end{itemize}
	\vspace{0.2cm}
\item Langage(s) de programmation
	\begin{enumerate}
	\item Travail sur les données (imputation) en Python
	\item Estimations en R
	\item[] $\Rightarrow$ transférabilité
	\end{enumerate}
	\vspace{0.2cm}
\item Articulation avec les autres modules (cf. mail)
	\begin{itemize}
	\item Effet de sélection?
	\end{itemize}

\vspace{0.2cm}
\item Travail sur les données côté CdC
		\begin{itemize}
		\item Imputation du grade en rétrospectif (matching)
		\item Correction des trajectoires aberrantes 
		\end{itemize}

\end{itemize}





\end{choixmarges}
\end{frame}




\begin{frame}
\frametitle{Points à discuter}
\framesubtitle{Simuler des réformes de grille}

\begin{choixmarges}{-0.5cm}{-0.5cm}


\begin{itemize}

\item Quels type de réformes envisagés?
	\begin{itemize}
	\item Allongement des grilles?
	\item Fusion de grilles? 
	\end{itemize}
	
\vspace{0.2cm}
\item Quel effet en simulation avec la modélisation adoptée? 

\vspace{0.2cm}
\item Effets non pris en compte:
	\begin{itemize}
	\item Paramètres non modélisés: condition de changement de corps, quotas de promotion fixés par decret, etc
	\item Endogénéité des comportements 
	\begin{itemize}
	\item Probabilité de changement de grade dépend de la structure des grilles
	\end{itemize}
	\end{itemize}

\end{itemize}


\end{choixmarges}
\end{frame}






\end{document}

% Glossaire
% dépendance: disability
