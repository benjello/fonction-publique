 
\documentclass[11pt,a4paper]{article}
\usepackage[english]{babel}
\usepackage[latin1]{inputenc}
\usepackage[OT1]{fontenc}
\usepackage{lmodern}
\usepackage{amsmath,amssymb, amsthm}
\usepackage{a4wide}
\usepackage[capposition=top]{floatrow}
\usepackage{verbatim}
\usepackage{float}
\usepackage{placeins}
\usepackage{flafter}
\usepackage{longtable}
\usepackage{import}
\usepackage{pdflscape}
\usepackage{rotating}
\usepackage{hhline}
\usepackage{multirow}
\usepackage{booktabs}
\usepackage[pdftex,pdfborder={0 0 0},colorlinks=true,linkcolor=blue,urlcolor=blue,citecolor=blue,bookmarksopen=true]{hyperref}
\usepackage{eurosym}
%\usepackage{breakcites}
\usepackage[autostyle]{csquotes}
%\usepackage{datetime}
\usepackage{natbib}
\usepackage{setspace}
\usepackage{lscape}
\usepackage[usenames]{color}
\usepackage{indentfirst}
\usepackage{url}
\usepackage{enumitem}
\usepackage{multirow}
\usepackage{subcaption}
\usepackage[justification=centering]{caption}
\bibliographystyle{agsm}

\usepackage{array}

\newcommand{\isEmbedded}{true}



\begin{document}

\title{Modeling the Career of the French Public Servants: \\ An Alternative to Wage Equations ?}


\author{Mahdi Ben Jelloul, Lisa Degalle and Simon Rabat\'e \thanks{Institut des politiques publiques \newline Contact: simon.rabate@ipp.eu}}


\maketitle

% Introduction
Dynamic microsimulation of pension system generally requires the modeling of labor market transitions and earnings trajectories. This article focuses on the earnings generating process of a specific population: the civil servants in the hospital and local public service in France. 

Accounting for the specificity of the population of interest, we develop an original approach 
based on the legislation-based salary scale. Compared to the private sector, work trajectories are much more predictable in the public sector: lay-off are merely possible, and earnings are, to a large extent, determined by the official salary structure. 

Official salary scales give the wage associated to each position in the scale, and the conditions required to move from a position to the other. We use those constraints to model earnings trajectories. Different types of model are estimated: multinomial logit for the ``choice'' (stay in the position on the scale, move to the next position in the scale, exit the scale) or hazard model for the duration spent in a given position and in a given scale. 

This works is part of a collaboration between the \textit{Caisse des d\'epots} and the Institute of public policy for the construction of a dynamic microsimulation for the pension scheme of  the civil servants in the hospital and local public service (CNRACL). 
For our estimations, we use a large panel dataset of the CNRACL pension scheme, providing employment and earning histories for the whole universe of the affiliates from years 2007 to 2015. 

We first provide graphical evidence of the relevance of the approach: the steps-wise profile of earning trajectories we observe cannot be straightforwardly predicted by the usual wage equations. To assess the validity of our results, we compare the prediction of our model to the observed trajectories for the last year of observation. We also compare our approach to more common wage equations to assess its efficiency.




\end{document}


