 
\documentclass[11pt,a4paper]{article}
\usepackage[utf8]{inputenc}		% LaTeX, comprend les accents !
\usepackage[T1]{fontenc}
\usepackage{natbib}	
%\usepackage[square,sort&compress,sectionbib]{natbib}		% Doit être chargé avant babel      
\usepackage[frenchb,english]{babel}
\usepackage{lmodern}
\usepackage{amsmath,amssymb, amsthm}
\usepackage{a4wide}
\usepackage[capposition=top]{floatrow}
\usepackage{verbatim}
\usepackage{float}
\usepackage{placeins}
\usepackage{flafter}
\usepackage{longtable}
\usepackage{import}
\usepackage{pdflscape}
\usepackage{rotating}
\usepackage{hhline}
\usepackage{multirow}
\usepackage{booktabs}
\usepackage[pdftex,pdfborder={0 0 0},colorlinks=true,linkcolor=blue,urlcolor=blue,citecolor=blue,bookmarksopen=true]{hyperref}
\usepackage{eurosym}
%\usepackage{breakcites}
\usepackage[autostyle]{csquotes}
%\usepackage{datetime}
\usepackage{natbib}
\usepackage{setspace}
\usepackage{lscape}
\usepackage[usenames]{color}
\usepackage{indentfirst}
\usepackage{url}
\usepackage{enumitem}
\usepackage{multirow}
\usepackage{subcaption}
\usepackage[justification=centering]{caption}
\bibliographystyle{agsm}

\usepackage{array}

\newcommand{\isEmbedded}{true}

\graphicspath{{../bordeaux/results/}}


\begin{document}

\selectlanguage{frenchb}
\title{Modélisation du changement de grade: premiers résultats}


\author{Mahdi Ben Jelloul, Lisa Degalle et Simon Rabat\'e}


\maketitle


La présente note est un support pour la présentation du jeudi 15 juin. Elle suit donc le cheminement des diapositives, tout en renvoyant à leur contenu pour la présentation des résultats. 


% Section I: principe général 
\section{Complétion de la durée dans le grade}
\subsection{Motivation}

On souhaite prédire le changement de grade de chaque agent. Ainsi, un changement de grade ne peut intervenir qu'une fois que l'agent a remplit des conditions de durée passée dans son grade. Ces durées minimales légales ouvrant l'accès au changement de grade varient entre 3 ans et 10 ans. Or, nous observons seulement le grade des agents sur une période de 5 ans, de 2011 à 2015. Cette fenêtre d'observation est trop étroite pour que l'on puisse identifier des agents de tous les grades ayant rempli la condition de durée dans le grade leur permettant de changer de grade. Il est donc nécessaire de mettre en place une procédure permettant d'imputer une durée dans le grade sur une plus grande période.\\
\indent D'autres conditions de changement de grade existent, telles que des conditions de franchissement d'échelon. Par exemple, le franchissement de l'échelon 5 dans le grade TTH2 est nécessaire pour accéder au grade TTH3. On impute donc un échelon à chaque agent pour chaque année entre 2002 et 2011.\\
\indent Ces conditions d'échelon et de durées passées dans le grade pourraient être plus ou moins déterminantes selon le grade précédent de chaque agent. On cherche donc à imputer à chaque agent, s'il y a lieu, le grade précédent l'entrée dans le grade courant.

\subsection{Sélection de l'échantillon}

\textit{Diapositive 3}\\

On sélectionne les agents qui sont Adjoints Techniques Territoriaux en 2011. Ce corps est retenu pour deux raisons : il est le corps le plus peuplé de la FPT et de la FPH et il est celui présentant la meilleure qualité de donnée. Seuls les agents présents dans ce corps pour la première année d'observations complètes, c.-à-d.~2011, sont retenus afin de réduire le nombre de cas pour lesquels il faudra inférer une position sur les grilles l'année précédente, c.-à-d.~2010. Sont exclus les agents nés avant 1960 afin de ne pas considérer les sorties en retraite. Ne sont également pas prises en considération les carrières des agents qui exhibent des incohérences sur une ou plusieurs périodes (code grade non renseigné alors que l'agent est en activité ou trajectoire de changement de grade non-autorisée).

\subsection{Description de la méthode d'imputation}

\textit{Diapositive 4}\\

On veut utiliser l'Indice Brut (IB) des agents avant 2011, ainsi que la dernière position connue de l'agent sur une grille (c.-à-d.~son corps, son grade et son échelon en 2011) afin de savoir si l'agent a changé de grade ou pas entre deux années consécutives (il est possible de raffiner en regardant des changements infra annuels), et donc identifier l'année d'entrée dans le grade de l'agent.
On s'efforcera donc de regarder si l'IB de chaque agent à t-1 est présent sur la grille du grade de l'agent à t, en prenant en compte d'éventuels changement de grille. Cette procédure est mise en oeuvre en Python.\\

Plus précisément :
\begin{itemize} 
	\item Si l'IB à t-1 de l'agent n'est pas présent sur la grille de son grade à t (en vigueur à t-1), on conclue directement que l'agent a changé de grade entre t-1 et t. On ne réitère pas la procédure sur cet agent, puisque seul le changement de grade nous intéresse. 
	
	\item Si l'IB de l'agent à t-1 est présent sur la grille de son grade à t et n'est pas présent sur la grille du grade précédent hiérarchiquement son grade à t, on conclue que l'agent n'a pas changé de grade entre t-1 et t. On réitère la procédure sur cet agent pour toutes les années où il ne change pas de grade. On fait une hypothèse ici en ne cherchant pas l'IB de l'agent sur la grille de l'ensemble des grades de tous les corps. On suppose que les transitions possibles dans les cas autres que le premier cas se font à l'intérieur du corps, et qu'il est uniquement possible de passer d'un grade au grade immédiatement supérieur à l'intérieur du corps.
	\item Si l'IB de l'agent à t-1 est présent sur la grille de son grade à t et sur une autre grille de son corps à t, on conclue que l'agent a changé de grade ou n'a pas changé de grade entre t-1 et t, on garde les deux cas possibles en attribuant à cette prédiction le statut "ambigu". On réitère la procédure sur les cas classés ambigus qui prédisent que l'agent reste dans son grade. On conserve l'hypothèse que dans tous les cas différents du premier cas, les transitions se font à l'intérieur du corps, et qu'il est uniquement possible de passer d'un grade au grade immédiatement supérieur à l'intérieur du corps.
\end{itemize}

Notons que si le grade de l'agent à t est le premier grade du corps (TTH1), aucune transition depuis le corps arrivant dans ce grade n'est possible. L'ensemble des agents arrivant en TTH1 entrent dans la fonction publique ou viennent d'un autre corps que celui des ATT. Si l'agent est à t dans un grade autre que TTH1, on détermine s'il vient du grade immédiatement hiérarchiquement inférieur à son grade à t ou s'il vient d'un autre corps.\\

Par ailleurs, notons que la procédure prend en compte des retards potentiels d'un an dans l'application de réformes des grilles (en particulier pour l'année 2006, année de création du corps des ATT).\\

Enfin, il importe de noter qu'il subsiste une incertitude sur le grade de certains agents-années. Cette incertitude est due aux chevauchements de grilles. En effet, un IB peut être à la fois présent sur une grille du grade TTH1 et sur une grille du grade TTH2 par exemple. Pour la grande majorité des agents, cette incertitude n'existe pas ou subsiste sur une ou deux années. Pour une minorité d'agents, il n'est pas possible de savoir en 2002 s'ils ont effectivement changé de grade entre 2002 et 2011 ou pas.\\

\subsection{Résultats}

\textit{Diapositives 5-12}\\

Comme expliqué dans la section précédente, les hypothèses faites sur les changements de grade nous permettent d'imputer une année d'entrée dans le grade de façon certaine dans la grande majorité des cas.\\

Les pics d'entrée dans le grade après un IB nul recoupent nos statistiques descriptives sur l'année d'affiliation des agents, même si certains pics d'entrée restent inexpliqués. Nous avons constaté une grande amélioration dans la régularité de ces graphiques après avoir pris en compte les retards potentiels dans l'application des réformes.

\subsection{Prochaines étapes}

On pourrait envisager de renouveller la procédure avec un pas trimestriel.
Par ailleurs, il serait bon de tester l'imputation en prenant comme année de départ l'année 2015 et en comparant entrées dans le grade prédites et observées entre 2011 et 2014.
Il s'agirait enfin de généraliser le programme afin de pouvoir compléter les carrières des agents d'autres corps que celui des ATT.

\section{L'impact des conditions institutionnelles: analyse graphique}

Nous illustrons l'impact des conditions de changement de grade sur les probabilités de changement de grade. Nous présentons l'effet des conditions séparément puis ensemble. La ligne représente le seuil institutionnel. 

\paragraph{TTH3:} Effet marqué et clair des conditions.

\paragraph{TTH2:} Pas d'effet de la condition de durée dans le grade, mais cela est logique car pour ce grade il s'agit d'une condition de durée passée dans le corps. Dont ne nous disposons malheureusement pas. Nous faisons l'approximation suivante: durée dans le corps = année courante - année d'affiliation. Ce faisant nous trouvons bien un effet de la condition de durée dans le corps, ainsi que de la condition d'échelon. 

\paragraph{TTH4:} Pas de pics, et pas de conditions, ce qui est cohérent. 
 
\paragraph{TTH1:} Les pics aux seuils n'apparaissent pas clairement. Pour la condition de grade, les pics apparaissent mieux quand on redéfinit l'année d'entrée dans le grade à partir de l'année d'affiliation. Cela pourrait suggérer un problème dans l'étape d'imputation. 
Pour la condition d'échelon le pic n'apparait pas non plus. Il est par ailleurs impossible de représenter les deux conditions sur le même plan comme pour les autres grades car il y a deux seuils (au choix, examen pro). 


\section{Estimation}

\subsection*{Modèles de durée:} 

Nous avions d'abord envisagé d'estimer la sortie de grade par modèle de durée. Nous avons rencontré plusieurs difficultés de fond, parmi lesquelles: 

\begin{enumerate}
\item Forte censure à droite (biais)
\item Troncature à gauche (sélection)
\item Estimations de modèle avec des \og time changing \fg{} et \og time varying \fg{} variables. 
\item Difficulté pour simuler les sorties de grade à partir des estimations. 
\end{enumerate}

Ces différentes raisons, en particulier la dernière, nous ont conduit à nous rabattre vers un modèle logit bivarié. Un retour sur les modèles de durée n'est pas exclu. 

\subsection*{Logit} 

Nous estimons un modèle logit dont la variable expliquée est la sortie du grade et les variables explicatives principales le grade, la durée dans le grade (simple et carré), et les indicatrices d'atteinte des conditions institutionnelles. 

Plusieurs spécifications sont testées, avec ou sans contrôles, effets de durée dans le grade ou effet des seuils institutionnels. 
L'effet est du sens attendu pour la promotion au choix, mais pas pour la promotion par examen professionnel. 

L'effet des seuils institutionnels semble important en magnitude.  % TODO pas clair significatif ? important ?  

\section{Simulation}


\subsection*{Test d'adéquation} 

Nous utilisons ensuite les coefficients estimés pour prédire les comportements. L'idée est de se rapprocher le plus possible d'un cadre de microsimulation. Le test utilisé à ce stade n'est sans doute pas satisfaisant: nous comparons les départs prédits et observés à l'année initiale 2011. 

Un test plus exigeant serait la simulation de toute la dynamique des sorties de grade entre 2011 et 2014. Mais pour cela, il faut modéliser l'évolution contrefactuelle des échelons, car l'échelon intervient dans les variables explicatives du modèle. Pour les individus qui sont sortis du grade en 2011, mais que la simulation ne fait pas partir en 2011, nous aurions besoin de re-calculer la probabilité de sortie pour cet individu en 2012. Une règle simple de modélisation serait l'application de la durée min/max/moyenne pour projeter. Mais cela nécessite une analyse plus fine de la durée effectivement passée en échelon au préalable. 

Diverses comparaisons entre départs observés et prédits sont implémentés: nombre moyen de départ, compte des erreurs et des bonnes prédictions, 

\subsection*{Résultats}

La prédiction donne des très bons résultats à la fois pour les modèles avec effet temporel et pour les modèles avec les variables de conditions. L'ajout des conditions n'améliore pas le fit du modèle. 

Toutefois, le test utilisé pour l'instant n'est sans doute pas assez exigeant.  

\section{Conclusion et prochaines étapes}

Voir diapositives. 


\end{document}


